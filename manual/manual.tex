\documentclass{article}

%% +-----------------------------------------------------------------+
%% | Packages                                                        |
%% +-----------------------------------------------------------------+

\usepackage[utf8]{inputenc}
\usepackage{xspace}
\usepackage{verbatim}
\usepackage[english]{babel}
\usepackage{amsmath}
\usepackage{amssymb}

%% +-----------------------------------------------------------------+
%% | Aliases                                                         |
%% +-----------------------------------------------------------------+

\newcommand{\obus}{\texttt{OBus}\xspace}
\newcommand{\dbus}{\texttt{D-Bus}\xspace}

%% +-----------------------------------------------------------------+
%% | Headers                                                         |
%% +-----------------------------------------------------------------+

\title{OBus user manual}
\author{Jérémie Dimino}

\begin{document}

\maketitle

\begin{abstract}
  D-Bus is an inter-processes communication protocol, or IPC for
  short, which is becomming a standard on desktop oriented
  computers. It is now possible to talk to a lot application using
  D-Bus. Moreover, it has many bindings/implementations for differents
  languages, which make it easily accessible. OBus is a pure OCaml
  implementation of it. What makes it different from other
  bindings/implementations is that it is the only one using
  cooperative threads, which make it very simple to fully exploit the
  asynchronous nature of the protocol.

  \textbf{Note:} it is advised to have some knowledge about the
  \texttt{lwt} library before reading this manual.
\end{abstract}

%% +-----------------------------------------------------------------+
%% | Table of contents                                               |
%% +-----------------------------------------------------------------+

\setcounter{tocdepth}{2}
\tableofcontents

%% +-----------------------------------------------------------------+
%% | Section                                                         |
%% +-----------------------------------------------------------------+
\section{Introduction}

OBus uses a syntax extension to ease its use, and make D-Bus method,
signal and property definitions more readable. Of course it is
possible not to use it. Code for both choices will be provided in this
manual.

\subsection{Overview of OBus}

\subsubsection{Packages}

OBus provides the folliwing packages (via findlib):

\begin{itemize}
\item ``\texttt{obus}'', which is the core library
\item ``\texttt{obus.syntax}'', which contains the syntax extension
\item ``\texttt{obus.hal}'', which is a binding to the Freedesktop
  Hal service
\item ``\texttt{obus.notification}'', which is a binding to the
  Freedesktop pop-up notifications service
\end{itemize}

\texttt{obus.hal} and \texttt{obus.notification} could be made outside
of OBus. They can be used as examples for writing bindings to D-Bus
services.

\subsubsection{Modules}

OBus contains 26 public modules. But do not be scared, most of the
time you will use a very small subset of them. The modules can be
divided in two categories:

\begin{itemize}
\item the high-level API
\item the low-level API
\end{itemize}

Here is a small example of what you need to write for using a D-Bus
service:

\begin{verbatim}
open OBus_pervasives

include OBus_interface.Make(struct let name = "some.interface" end)

OP_method Foo : int -> int
OP_method Bar : string
OP_signal Plop : string * string

lwt () =
  lwt bus = Lazy.force OBus_bus.session in
  let proxy = Obus_proxy.make ~peer:(OBus_peer.make bus "some.destination") ~path:["some"; "path"] in
  lwt x = foo proxy in
  lwt y = bar proxy in
  ...
\end{verbatim}

We will demystify the meanings of all this in the next sections.

\subsection{How to read this manual}

All example given can be executed in an ocaml toplevel. The first
things you have to do is to start one and load the OBus library:

\begin{verbatim}
# #use "topfind";;
# #require "obus";;
\end{verbatim}

And if you want to try the syntax extension:

\begin{verbatim}
# #require "obus.syntax";;
\end{verbatim}

You may also load the Lwt syntax extension, as examples of this manual
use it:

\begin{verbatim}
# #require "lwt.syntax";;
\end{verbatim}

%% +-----------------------------------------------------------------+
%% | Section                                                         |
%% +-----------------------------------------------------------------+
\section{Basis}

\subsection{Connections and message buses}

There is two way to talk to another application speaking the D-Bus
protocol: via a direct connection to the application or via a special
application called a message bus. A message bus act as a router
between several applications. On a desktop computer, there are two
well-known instances: the system message bus, and the user session
message bus.

The first one is unique given a computer, and use security
policies. The second is unique given a user session. Its goal is to
allow programs running in the session to talk to each other.

OBus offers have predefined connections for connecting to these
message buses:

\begin{verbatim}
# OBus_bus.session;;
- : OBus_bus.t Lwt.t Lazy.t = <lazy>
# OBus_bus.system;;
- : OBus_bus.t Lwt.t Lazy.t = <lazy>
\end{verbatim}

As you can see, both are lazy threads returing a message bus
object. Let's connect to the message bus:

\begin{verbatim}
# lwt bus = Lazy.force OBus_bus.session;;
val bus : OBus_bus.t = <abstr>
\end{verbatim}

Now, we have a connection to the session message bus. We try some of
the methods of the message bus itself:

\begin{verbatim}
# lwt id = OBus_bus.get_id bus;;
val id : OBus_uuid.t = <abstr>
# OBus_uuid.to_string id;;
- : string = "45364dda6ce1fbcfaa9484754b5da696"
\end{verbatim}

\subsection{D-Bus names}

\subsection{D-Bus addresses}

\subsection{Vocabulary}

% src/OBus_address
% src/OBus_bus
% src/OBus_connection

%% +-----------------------------------------------------------------+
%% | Section                                                         |
%% +-----------------------------------------------------------------+
\section{The OBus type system}

\subsection{D-Bus types and values}



OCaml representation of D-Bus values are given by the
\texttt{OBus\_value} module:

\begin{verbatim}
type tbasic =
  | Tbyte
  | Tboolean
  | Tint16
  | Tint32
  | Tint64
  | Tuint16
  | Tuint32
  | Tuint64
  | Tdouble
  | Tstring
  | Tsignature
  | Tobject_path
  | Tunix_fd

type tsingle =
  | Tbasic of tbasic
  | Tstructure of tsingle list
  | Tarray of tsingle
  | Tdict of tbasic * tsingle
  | Tvariant

type tsequence = tsingle list
\end{verbatim}

D-Bus has its own system of types and values.

D-Bus basic types are \texttt{BYTE}, \texttt{BOOLEAN}, \texttt{INT8},
\texttt{UINT8}, \texttt{INT16}, \texttt{UINT16}, \texttt{INT32},
\texttt{UINT32}, \texttt{INT64}, \texttt{UINT64}, \texttt{DOUBLE},
\texttt{STRING}, \texttt{SIGNATURE}, \texttt{OBJECT\_PATH} and
\texttt{UNIX\_FD}.

D-Bus container types are \texttt{ARRAY}, \texttt{DICT\_ENTRY},
\texttt{STRUCTURE} and \texttt{VARIANT}

\subsection{OBus type combinators}

OBus type combinators are a way to convert OCaml values into D-Bus
values and from D-Bus types to OCaml values.

% src/OBus_type
% src/OBus_pervasives

\begin{center}
  \begin{tabular}{|c|c|c|}
    \hline
    \textbf{OBus type combinator} & \textbf{D-Bus type} & \textbf{Caml type} \\
    \hline
    \texttt{char, byte} & \texttt{BYTE} & \texttt{char} \\
    \hline
    \texttt{bool, boolean} & \texttt{BOOLEAN} & \texttt{bool} \\
    \hline
    \texttt{int8} & \texttt{INT8} & \texttt{int} \\
    \hline
    \texttt{uint8} & \texttt{UINT8} & \texttt{int} \\
    \hline
    \texttt{int16} & \texttt{INT16} & \texttt{int} \\
    \hline
    \texttt{uint16} & \texttt{UINT16} & \texttt{int} \\
    \hline
    \texttt{int32} & \texttt{INT32} & \texttt{int32} \\
    \hline
    \texttt{uint32} & \texttt{UINT32} & \texttt{int32} \\
    \hline
    \texttt{int64} & \texttt{INT64} & \texttt{int64} \\
    \hline
    \texttt{uint64} & \texttt{UINT64} & \texttt{int64} \\
    \hline
    \texttt{int} & \texttt{INT32} & \texttt{int} \\
    \hline
    \texttt{uint} & \texttt{UINT32} & \texttt{int} \\
    \hline
    \texttt{float, double} & \texttt{DOUBLE} & \texttt{float} \\
    \hline
    \texttt{string} & \texttt{STRING} & \texttt{string} \\
    \hline
    \texttt{signature} & \texttt{SIGNATURE} & \texttt{OBus\_value.signature} \\
    \hline
    \texttt{path, object\_path} & \texttt{OBJECT\_PATH} & \texttt{OBus\_path.t} \\
    \hline
    \texttt{file\_descr} & \texttt{UNIX\_FD} & \texttt{Lwt\_unix.file\_descr} \\
    \hline
    \texttt{unix\_file\_descr} & \texttt{UNIX\_FD} & \texttt{Unix.file\_descr} \\
    \hline
    \texttt{$'elt$ list} & \texttt{ARRAY of $elt$} & \texttt{$'elt$ list} \\
    \hline
    \texttt{$'elt$ array} & \texttt{ARRAY of $elt$} & \texttt{$'elt$ array} \\
    \hline
    \texttt{byte\_array} & \texttt{ARRAY of byte} & \texttt{string} \\
    \hline
    \texttt{($'key$, $'value$) dict} & \texttt{ARRAY of DICT\_ENTRY of $key$ and $value$} & \texttt{($'key$ * $'value$) list} \\
    \hline
    \texttt{$'elts$ structure} & \texttt{STRUCTURE of $elts$} & \texttt{$'elts$} \\
    \hline
    \texttt{variant} & \texttt{VARIANT} & \texttt{OBus\_value.single} \\
    \hline
    \texttt{unit} & \textit{(empty sequence)} & \texttt{unit} \\
    \hline
  \end{tabular}
\end{center}

%% +-----------------------------------------------------------------+
%% | Section                                                         |
%% +-----------------------------------------------------------------+
\section{Writing bindings for D-Bus services}

% src/OBus_interface
% src/OBus_object

%% +-----------------------------------------------------------------+
%% | Section                                                         |
%% +-----------------------------------------------------------------+
\section{Tools}

\subsection{obus-dump}
\subsection{obus-introspect}
\subsection{obus-binder}

%% +-----------------------------------------------------------------+
%% | Section                                                         |
%% +-----------------------------------------------------------------+
\section{Launching a D-Bus server}

% src/OBus_server

%% +-----------------------------------------------------------------+
%% | Section                                                         |
%% +-----------------------------------------------------------------+
\section{Advanced}

% src/OBus_error

% src/OBus_info
% src/OBus_introspect
% src/OBus_message
% src/OBus_name
% src/OBus_path
% src/OBus_peer
% src/OBus_proxy
% src/OBus_resolver
% src/OBus_string
% src/OBus_uuid

% src/OBus_match

% src/OBus_wire
% src/OBus_auth
% src/OBus_transport

\end{document}
